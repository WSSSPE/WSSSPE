%%%%%%%%%%%%%%%%%%%%%%%%%%%%%%%%%%%%%%%%%%%%%%%%%%%%%%%%%%%%
\subsection{White paper on developing sustainable software}
\label{sec:best-practices-developing}
%%%%%%%%%%%%%%%%%%%%%%%%%%%%%%%%%%%%%%%%%%%%%%%%%%%%%%%%%%%%

\note{Sandra to write this}

\note{Introduction to group here, including the overall objective of work in this area.}

Devloping sustainable software can be investigated under many diverse aspects and dimensions: 
economic, technical, environmental and social are only a few. The white paper will focus 
on scientific environments and their implications and the target group are especially developer 
and project manager of scientific software. Due to the complexity of this field it is important to select
a subset of sustainability aspects for the white paper. The idea is to create a series of papers instead of trying
to tackle all important topics in one paper. For the first white paper, we aim to set the stage with successful use cases and analyze why they have been successful. Further topics include community-related practices, government and management, funding, metrics, tools and usability. We will collect feedback from the WSSSPE community on the white paper and extend it to a journal paper.

\subsubsection{Participants}

Jeffrey C. Carver <carver@cs.ua.edu>,
Neil Chue Hong <N.ChueHong@software.ac.uk>,
Tom Crick <tcrick@cardiffmet.ac.uk>,
Miguel de~Val-Borro <valborro@princeton.edu>,
Hans Fangohr <fangohr@soton.ac.uk>
Sandra Gesing <sandra.gesing@nd.edu>,
Derek Groen <Derek.Groen@brunel.ac.uk>,
Dan Gunter <DKGunter@lbl.gov>,
Daniel S. Katz <d.katz@ieee.org>,
Alexander Konovalov <alexander.konovalov@st-andrews.ac.uk>,
Frank L\"offler <knarf@cct.lsu.edu>,
Suresh Marru <smarru@iu.edu>,
Kyle E. Niemeyer <kyle.niemeyer@oregonstate.edu>,
Abani Patra <abani@buffalo.edu>
and
Francisco Queiroz <chico@puc-rio.br>

\subsubsection{Working group objective}

\note{Specific things the working group wants to accomplish in the context of the larger objective.}

The working group aims at starting a series of papers and support developers and project managers of scientific software. While there are already a few papers available on best practices and sustainability of scientific software in general, the goal is to achieve a series of papers that leads to consensus in the community, tackle many diverse aspects and stays up-to-date with novel trends. The goal is quite ambitious and we hope to attract more authors over time who would like to contribute to specific aspects and take the lead on those. 

\subsubsection{Gap or challenge}

The challenge for the group is to accomplish a first version of the white paper to kick off the series. The complexity of the topic has led in 2016 to be a bit too ambitious with trying to cover a wide range of topics. The distribution of topics over time seems to be a better strategy to get a paper finalized. The series would fill also a gap on best practices that the community can contribute to.

\subsubsection{Relevant people and resources}

The list of contributors, as of November 10th 2016, includes (in alphabetical order): \\
Jeffrey C. Carver <carver@cs.ua.edu>,
Neil Chue Hong <N.ChueHong@software.ac.uk>,
Tom Crick <tcrick@cardiffmet.ac.uk>,
Miguel de~Val-Borro <valborro@princeton.edu>,
Hans Fangohr <fangohr@soton.ac.uk>
Sandra Gesing <sandra.gesing@nd.edu>,
Derek Groen <Derek.Groen@brunel.ac.uk>,
Dan Gunter <DKGunter@lbl.gov>,
Daniel S. Katz <d.katz@ieee.org>,
Alexander Konovalov <alexander.konovalov@st-andrews.ac.uk>,
Frank L\"offler <knarf@cct.lsu.edu>,
Suresh Marru <smarru@iu.edu>,
Kyle E. Niemeyer <kyle.niemeyer@oregonstate.edu>,
Abani Patra <abani@buffalo.edu>
and
Francisco Queiroz <chico@puc-rio.br>

\subsubsection{Plans}

\note{What tasks will the working group undertake}
The resulting white paper will be distributed via the WSSSPE email list. Via collecting feedback on it, the further plans are to attract a wider community that contributes to the journal paper and investigate whether more people would like to be involved in the paper series.

\subsubsection{SMART steps}

First eight SMART steps proposed: \\
\begin{itemize}
\item List existing contributors and open GitHub for new ones - Sandra Gesing (done).
\item Establish preliminary timeline  - Sandra Gesing (done).
\item Organize related work - Francisco Queiroz (done).
\item Define scope of the paper - Sandra Gesing (done).
\item Suggest new sections - Francisco Queiroz (done).
\item Define leading authors and contributors for sections - Francisco Queiroz (done).
\item Finalize first version - Sandra Gesing (in progress)
\item Distribute to WSSSPE community and collect feedback - Sandra Gesing (in progress)
\end{itemize}

\subsubsection{More information \& joining instructions}

The GitHub repository for the White Paper can be found at \url{https://github.com/WSSSPE/WG-Best-Practices}. For more information and requests, join the \texttt{\#wg-best-practices} channel at WSSSPE's Slack team.
