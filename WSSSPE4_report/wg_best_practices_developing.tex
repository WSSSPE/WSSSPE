%%%%%%%%%%%%%%%%%%%%%%%%%%%%%%%%%%%%%%%%%%%%%%%%%%%%%%%%%%%%
\subsection{White paper on developing sustainable software}
\label{sec:best-practices-developing}
%%%%%%%%%%%%%%%%%%%%%%%%%%%%%%%%%%%%%%%%%%%%%%%%%%%%%%%%%%%%

%\note{Sandra to write this}

%\note{Introduction to group here, including the overall objective of work in this area.}

Many diverse aspects and dimensions of developing sustainable software can be investigated,
such as economic, technical, environmental, and social. This group aims writing white papers that
will focus 
on scientific environments and their implications, targeted at developers 
and project managers of scientific software. Given the complexity of this field, it is important to select
a subset of sustainability aspects for the white paper. The idea is to create a series of papers instead of trying
to tackle all important topics in one paper. For the first white paper, the group aims to set the stage with successful use cases and an analysis of why they have been successful. Further topics will include community-related practices, government and management, funding, metrics, tools and usability. The group will collect feedback from the WSSSPE community on this white paper and extend it to a journal paper.

\subsubsection{Participants}

\begin{itemize}
\item Jeffrey C. Carver <carver@cs.ua.edu>,
\item Neil Chue Hong <N.ChueHong@software.ac.uk>,
\item Tom Crick <tcrick@cardiffmet.ac.uk>,
\item Miguel de~Val-Borro <valborro@princeton.edu>,
\item Hans Fangohr <fangohr@soton.ac.uk>
\item Sandra Gesing <sandra.gesing@nd.edu>,
\item Derek Groen <Derek.Groen@brunel.ac.uk>,
\item Dan Gunter <DKGunter@lbl.gov>,
\item Daniel S. Katz <d.katz@ieee.org>,
\item Alexander Konovalov <alexander.konovalov@st-andrews.ac.uk>,
\item Frank L\"offler <knarf@cct.lsu.edu>,
\item Suresh Marru <smarru@iu.edu>,
\item Kyle E. Niemeyer <kyle.niemeyer@oregonstate.edu>,
\item Abani Patra <abani@buffalo.edu>
\item Francisco Queiroz <chico@puc-rio.br>
\end{itemize}

\subsubsection{Working group objective}

%\note{Specific things the working group wants to accomplish in the context of the larger objective.}

The working group aims at starting a series of papers and supporting developers and project managers of scientific software. While there are already a few papers available on best practices and sustainability of scientific software in general, the group's goal is to create a series of papers that lead to consensus in the community, tackle many diverse aspects, and stay up-to-date with new trends. This goal is quite ambitious, and the group hopes to attract more authors over time who would like to contribute to specific aspects and take the lead on them. 

\subsubsection{Gap or challenge}

The challenge for the group is to accomplish a first version of the white paper to kick off the series. The complexity of the topic has led to a first attempt in 2016 that may have been a bit too ambitious, trying to cover a wide range of topics. The addressing of different topics over time seems to be a better strategy to finalize a paper. The series would fill also a gap on best practices to which the community can contribute.

\subsubsection{Relevant people and resources}

The list of contributors, as of November 10th 2016, is the same as the list of participants above.

\subsubsection{Plans}

%\note{What tasks will the working group undertake}
The resulting draft white paper will be distributed via the WSSSPE email list. After collecting feedback on it, the group's plans are to attract a wider community to contribute to an extended journal paper version and to investigate whether more people would like to be involved in the white paper series.

\subsubsection{SMART steps}

First eight SMART steps proposed: \\
\begin{itemize}
\item List existing contributors and open GitHub for new ones -- Sandra Gesing (done)
\item Establish preliminary timeline  -- Sandra Gesing (done)
\item Organize related work -- Francisco Queiroz (done)
\item Define scope of the paper -- Sandra Gesing (done)
\item Suggest new sections -- Francisco Queiroz (done)
\item Define leading authors and contributors for sections -- Francisco Queiroz (done)
\item Finalize first version -- Sandra Gesing (in progress)
\item Distribute to WSSSPE community and collect feedback -- Sandra Gesing (in progress)
\end{itemize}

\subsubsection{More information \& joining instructions}

The GitHub repository for the white paper can be found at \url{https://github.com/WSSSPE/WG-Best-Practices}. For more information and requests, join the \texttt{\#wg-best-practices} channel at WSSSPE's Slack team.
