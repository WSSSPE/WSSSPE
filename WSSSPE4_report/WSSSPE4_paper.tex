\documentclass[11pt, oneside]{amsart}
\pdfoutput=1

\usepackage{amsmath,amssymb,amsmath}

\usepackage{color}
\usepackage[table]{xcolor}
\usepackage{dcolumn}
\usepackage{caption}
\usepackage{graphicx}

\usepackage[T1]{fontenc}
\usepackage[utf8]{inputenc}
\usepackage{lmodern}
\usepackage[font=small]{caption}
\usepackage{varwidth}

\usepackage[activate={true,nocompatibility},final,tracking=true,kerning=true,spacing=true,factor=1100,stretch=10,shrink=10]{microtype}
\microtypecontext{spacing=nonfrench}
\usepackage{xspace}

\usepackage{rotating}
\usepackage{caption,subcaption}
\usepackage{float}
\usepackage{psfrag}
\usepackage{tabularx}
\usepackage[hyphens]{url}
\usepackage{wrapfig}
\usepackage{longtable}
\usepackage{verbatim}
\usepackage{array,booktabs,multicol,multirow}

%better printing of numbers
\usepackage[english]{babel}
\usepackage{textcomp}
\usepackage{csquotes}

% The following three lines are used for displaying footnote in tables.
\usepackage{footnote}
\makesavenoteenv{tabular}
\makesavenoteenv{table}

\usepackage{placeins}

\usepackage{enumitem}
\setlist{leftmargin=7mm}

%\setcounter{secnumdepth}{3}
%\setcounter{tocdepth}{3}

\usepackage[bookmarks, bookmarksopen, bookmarksnumbered, colorlinks,linkcolor=blue,urlcolor=blue,citecolor=blue]{hyperref}
\usepackage[all]{hypcap}
\urlstyle{rm}

\definecolor{orange}{rgb}{1.0,0.3,0.0}
\definecolor{violet}{rgb}{0.75,0,1}
\definecolor{darkgreen}{rgb}{0,0.6,0}
\definecolor{cyan}{rgb}{0.2,0.7,0.7}
\definecolor{blueish}{rgb}{0.2,0.2,0.8}
\definecolor{darkblue}{rgb}{0.1,0.1,0.9}
\definecolor{lightgray}{gray}{0.9}

\newcommand{\todo}[1]{{\color{blue}$\blacksquare$~\textsf{[TODO: #1]}}}
\newcommand{\note}[1]{ {\textcolor{blueish}    { ***Note:      #1 }}}
\newcommand{\katznote}[1]{ {\textcolor{magenta}    { ***Dan:      #1 }}}
\newcommand{\gabnote}[1]{ {\textcolor{cyan}    { ***Gabrielle:     #1 }}}
\newcommand{\nchnote}[1]{  {\textcolor{orange}      { ***Neil: #1 }}}
\newcommand{\kylenote}[1]{  {\textcolor{cyan}      { ***Kyle: #1 }}}

% Don't use tt font for urls
\urlstyle{rm}

\usepackage[letterpaper, margin=1in]{geometry}
% You can use a baselinestretch of down to 0.9
\renewcommand{\baselinestretch}{0.96}

\sloppypar

\begin{document}

\title[]{Fourth Workshop on Sustainable Software for Science: Practice and Experiences (WSSSPE4)}

\author{Daniel~S.~Katz$^{(1)}$,
Kyle~E.~Niemeyer$^{(2)}$,
Sandra~Gesing$^{(3)}$,
Lorraine~Hwang$^{(4)}$,
Wolfgang~Bangerth$^{(5)}$,
Simon~Hettrick$^{(6)}$,
Ray Idaszak$^{(7)}$,
Jean~Salac$^{(8)}$,
Neil~Chue~Hong$^{(9)}$,
Santiago~N\'u\~nez-Corrales$^{(10)}$,
Alice~Allen$^{(11)}$,
R.~Stuart~Geiger$^{(12)}$,
Jonah~Miller$^{(13)}$,
Emily~Chen$^{(14)}$,
Anshu~Dubey$^{(15)}$,
Patricia~Lago$^{(16)}$
%\\Author order to be determined based on contributions
}

%
\thanks{{}$^{(1)}$ National Center for Supercomputing Applications (NCSA) \&
Department of Computer Science  \&
Department of Electrical and Computer Engineering  \&
School of Information Sciences (iSchool),
University of Illinois, Urbana--Champaign, IL, USA; d.katz@ieee.org; ORCID: 0000-0001-5934-7525}
%
\thanks{{}$^{(2)}$ School of Mechanical, Industrial, and Manufacturing Engineering,
Oregon State University, Corvallis, OR, USA; kyle.niemeyer@oregonstate.edu; ORCID: 0000-0003-4425-7097}
%
\thanks{{}$^{(3)}$ Center for Research Computing \& Department of Computer Science and Engineering,
University of Notre Dame, IN, USA; sandra.gesing@nd.edu; ORCID: 0000-0002-6051-0673}
%
\thanks{{}$^{(4)}$ University of California, Davis, CA, USA; ljhwang@ucdavis.edu; ORCID: 0000-0002-1021-3101}
%
\thanks{{}$^{(5)}$ Department of Mathematics, Colorado State
  University, Fort Collins, CO, USA; bangerth@colostate.edu; ORCID: 0000-0003-2311-9402}
%
\thanks{{}$^{(6)}$ The Software Sustainability Institute, Southampton, UK; s.hettrick@software.ac.uk; ORCID: 0000-0002-6809-5195}
%
\thanks{{}$^{(7)}$ RENCI, University of North Carolina at Chapel Hill, Chapel Hill, NC, USA; rayi@renci.org; ORCID: 0000-0002-3444-7615}
%
\thanks{{}$^{(8)}$ University of Virginia, Charlottesville, VA, USA; jeansalac@virginia.edu; ORCID: 0000-0001-5971-9333}
%
\thanks{{}$^{(9)}$ University of Edinburgh, UK; N.ChueHong@epcc.ed.ac.uk;
ORCID: 0000-0002-8876-7606}
%
\thanks{{}$^{(10)}$ Illinois Informatics Institute (I3) and National Center for Supercomputing Applications (NCSA), University of Illinois at Urbana-Champaign, IL, USA;
nunezco2@illinois.edu;
ORCID: 0000-0003-4342-6223}
%
\thanks{{}$^{(11)}$ Astrophysics Source Code Library, Astronomy Department, University of Maryland, College Park, MD, USA; aallen@ascl.net; ORCID: 0000-0003-3477-2845}
%
\thanks{{}$^{(12)}$ University of California, Berkeley, Berkeley Institute for Data Science (BIDS), Berkeley, CA, USA; stuart@stuartgeiger.com; ORCID: 0000-0001-7215-0532}
%
\thanks{{}$^{(13)}$ Perimeter Institute for Theoretical Physics \&
Department of Physics, University of Guelph,
Guelph, ON, Canada;
jmiller@perimeterinstitute.ca; ORCID: 0000-0001-6432-7860}
%
\thanks{{}$^{(14)}$ University of Illinois at Urbana-Champaign, USA; echen35@illinois.edu; ORCID: 0000-0003-2578-0296}
%
\thanks{{}$^{(15)}$ Mathematics and Computer Science Division,
Argonne National Laboratory,
Lemont, IL, USA \&
University of Chicago,
Chicago, IL, USA;
Adubey@anl.gov;
ORCID: 0000-0003-3299-7426}
%
\thanks{{}$^{(16)}$ Vrije Universiteit Amsterdam, The Netherlands; p.lago@vu.nl; ORCID: 0000-0002-2234-0845}


\begin{abstract}

This article summarizes motivations, organization, and activities of the Fourth Workshop on Sustainable Software for Science: Practice and Experiences (WSSSPE4). The WSSSPE series promotes sustainable research software by positively impacting principles and best practices, careers, learning, and credit. This article includes the mission and vision statements that were drafted at the workshop and finalized shortly after it; the keynote and idea papers, position papers, experience papers, demos, and lightning talks presented during the workshop; and a panel discussion on best practices. The main part of the article discusses the set of working groups that formed during the meeting, along with contact information for readers who may want to join a group. Finally, it discusses a survey of the workshop attendees.

% It also summarizes a set of lightning
% talks in which speakers highlighted to-the-point lessons and challenges
% pertaining to sustaining scientific software.
% The final and main contribution of the report is a summary of the
% discussions, future steps, and future organization for a set of self-organized
% working groups on topics including developing pathways to funding scientific
% software; constructing useful common metrics for crediting software
% stakeholders; identifying principles for sustainable software engineering
% design; reaching out to research software organizations around the world; and
% building communities for software sustainability. For each group, we include a
% point of contact and a landing page that can be used by those who want to join
% that group's future activities. The main challenge left by the workshop is to
% see if the groups will execute these activities that they have scheduled, and
% how the WSSSPE community can encourage this to happen.

\end{abstract}



\maketitle
\newpage

%%%%%%%%%%%%%%%%%%%%%%%%%%%%%%%%%%%%%%%%%%%%%%%%%%%%%%%%%%%%
\section{Introduction} \label{sec:intro}
%%%%%%%%%%%%%%%%%%%%%%%%%%%%%%%%%%%%%%%%%%%%%%%%%%%%%%%%%%%%


The Fourth Workshop on Sustainable Software for Science: Practice and Experiences
(WSSSPE4)\footnote{\url{http://wssspe.researchcomputing.org.uk/wssspe4/}} was
held over 2 1/2 days on 12--14 September 2016 in Manchester, England, with 68 attendees.
This location and date was selected so that WSSSPE4 immediately preceded the
First Research Software Engineers (RSE) Conference.
Previous events in the WSSSPE series include
WSSSPE1\footnote{\url{http://wssspe.researchcomputing.org.uk/wssspe1/}}~\cite{WSSSPE1-pre-report,WSSSPE1},
held in conjunction with the SC13 conference;
WSSSPE1.1\footnote{\url{http://wssspe.researchcomputing.org.uk/wssspe1-1/}}, a
focused workshop organized jointly with the SciPy
conference\footnote{\url{https://conference.scipy.org/scipy2014/participate/wssspe/}};
WSSSPE2\footnote{\url{http://wssspe.researchcomputing.org.uk/wssspe2/}}~\cite{WSSSPE2-pre-report,WSSSPE2},
held in conjunction with SC14;
WSSSPE2.1\footnote{\url{http://wssspe.researchcomputing.org.uk/wssspe2-1/}}, a
focused workshop organized again jointly with
SciPy\footnote{\url{http://scipy2015.scipy.org/ehome/115969/286469/}};
and WSSSPE3\footnote{\url{http://wssspe.researchcomputing.org.uk/wssspe3/}}~\cite{WSSSPE3},
held in Boulder, Colorado, USA.
The first WSSSPE workshop was named
``Working towards
Sustainable Software for Science: Practice and Experiences,'' which remains the meaning
of the WSSSPE group, but the workshops after that were named
``Workshop on Sustainable
Software for Science: Practice and Experiences.'' Together these reflect
that WSSSPE is both a community and a set of workshops.

%\todo{Introduction}

The WSSSPE4 workshop included multiple mechanisms for participation and
encouraged team building around solutions. WSSSPE4 strongly encouraged participation
of early-career scientists, postdoctoral researchers, graduate students,
early-stage researchers, and those from underrepresented groups,
with funds provided to the conference organizers by the National Science
Foundation (NSF), the Gordon and Betty Moore Foundation, the Alfred P.~Sloan Foundation, and the Software
Sustainability Institute (SSI) to support the travel of potential participants
who would not otherwise be able to attend the workshop. These
funds allowed 29 additional people % to attend and participate. A subset of the
% organizing committee reviewed 
out of 44 applicants which 
% ions for travel support and
% competitively selected the awardees, 
included 11 students and five early-career
researchers.
%  In addition, the travel award subcommittee tried to increase the
% diversity of applicants by forwarding the notice of travel support to
% organizations including PyLadies\footnote{\url{http://www.pyladies.com}},
% Django Girls\footnote{\url{https://djangogirls.org}},
% Women Who Code\footnote{\url{https://www.womenwhocode.com}}, and
% Women in HPC\footnote{\url{http://www.womeninhpc.org}}.
%\todo{did we reach out to any others?}
%
WSSSPE4 also included two professional event organizers/facilitators who helped
plan the workshop agenda,
and who engaged with participants during the workshop.

At the workshop, we introduced a Code of Conduct (CoC).%
\footnote{\url{http://wssspe.researchcomputing.org.uk/wssspe4/code-of-conduct/}}
The CoC was conceived for the workshop and a guideline for the community of scientists that WSSSPE
supports, including their personal and online interactions (e.g., on
Twitter, in email lists, in the Slack team). The WSSSPE4 CoC is based on the
FORCE11 conference CoC~\cite{FORCE11:CoC}, which is in turn based on the Code4Lib
CoC~\cite{Code4Lib:CoC}.
The main guidelines of the CoC are:
\begin{quote}
    WSSSPE events are community events intended for networking and collaboration
    as well as learning. We value the participation of every member of the
    community and want all attendees to have an enjoyable and fulfilling
    experience. Accordingly, all attendees are expected to show respect and
    courtesy to other attendees throughout the event and in interactions online
    associated with the event.

    The WSSSPE event organizers are dedicated to providing a harassment-free
    experience for everyone, regardless of gender, gender identity and
    expression, age, sexual orientation, disability, physical appearance,
    body size, race, ethnicity, religion (or lack thereof), technology choices,
    or other group status.

    To make clear what is expected, everyone taking part in WSSSPE events and
    discussions---speakers, helpers, organizers, and participants---is required
    to conform to the Code of Conduct, listed on the community website \cite{}.

%     \begin{itemize}
%     \item Communicate appropriately for a professional audience including
%     people of many different backgrounds. Sexual language and imagery are not
%     appropriate for any event.

%     \item Be kind to others. Do not insult or put down other attendees. Be
%     mindful of jargon, which can sometimes exclude others from engaging in the
%     discussion.

%     \item Behave professionally. Remember that harassment and sexist, racist,
%     ageist, or exclusionary behavior are not appropriate.
%     \end{itemize}
 \end{quote}

The CoC was introduced at the beginning of WSSSPE4, with the CoC subcommittee
and a general email address for reporting concerns or incidents, or
asking questions.  One concern was raised after the first half day, 
% regarding how part of the workshop was being run, 
and we changed the workshop
to address this.

This paper is a summary of a longer report~\cite{WSSSPE4-report} on the events at the workshop and the submitted materials.
The remainder of the paper includes the call for papers (\S\ref{sec:preworkshop}); the WSSSPE mission and vision (\S\ref{sec:mission}); a set of presentations that included a keynote, papers, lighting talks, and a panel (\S\ref{sec:presentations}); and the activities of a set of working groups (\S\ref{sec:WGs}).  One full day of the
workshop was spent with participants in the working groups, which occurred in parallel
with each other.  And each of the working groups left with a plan for how they could move
forward.
This report also mentions the Slack channel created for further discussions (\S\ref{sec:slack}), and it highlights an attendee survey (\S\ref{sec:survey}) before concluding (\S\ref{sec:conclusions}).
An appendix lists the organizers and program committee members (\S\ref{sec:orgcom}).



%%%%%%%%%%%%%%%%%%%%%%%%%%%%%%%%%%%%%%%%%%%%%%%%%%%%%%%%%%%%
\section{Call for participation} \label{sec:preworkshop}
%%%%%%%%%%%%%%%%%%%%%%%%%%%%%%%%%%%%%%%%%%%%%%%%%%%%%%%%%%%%

WSSSPE4 was based on the work done at previous WSSSPE meetings, but aimed
at producing working groups that better continued working after the workshop ended.
In addition, based on feedback after WSSSPE3, it became clear that attendees
had two distinct motivations:  to make a better future
for research software, and to immediately do better research software development.
This led to the idea of WSSSPE4 being partially divided into two tracks:

\begin{quote}
    \textbf{Track 1 -- Building a sustainable future for open-use research
    software}: defining a vision of the future of open-use
    research software, and in the workshop, initiating the activities that are
    needed to get there.

    \noindent \textbf{Track 2 -- Practices \& experiences in sustainable scientific software}:
    improving the quality of today's research software and the
    experiences of its developers by sharing best practices and experiences.
\end{quote}

The call for participation requested:
idea papers, implementable proposals related to Track~1;
position papers, longer, not previously published papers
discussing what we can do to improve sustainable scientific
software in the short term;
experience papers, longer papers that discuss current
experiences and how they have been used to improve the quality of
today's research software and/or the experiences of its developers;
demos, brief papers describing a tool or
process that would be demonstrated that improves the quality of today's research
software and\slash or the experiences of its developers; and
extended abstracts describing lightning talks,
where each author could make a brief statement about work that either had been
done or was needed.

Suggested contribution topics included:

\begin{itemize}
\renewcommand{\labelenumi}{\textbf{\theenumi}.}
\setlength{\rightmargin}{1em}

\item Development and Community:
best practices for developing sustainable software;
models for funding specialist expertise in software collaborations;
software tools that aid sustainability;
academia/industry interaction;
refactoring/improving legacy scientific software;
engineering design for sustainable software;
metrics for the success of scientific software; and
adaptation of mainstream software practices for scientific software

\item Professionalization:
career paths;
RSE as a brand;
RSE outside of the UK or Europe; and
Increase incentives in publishing, funding and promotion for better software

\item Training:
taining for developing sustainable software; and 
curriculum for software sustainability

\item Credit:
making the existing credit and citation ecosystem work better for software;
future credit and citation ecosystem;
software contributions as a part of tenure review;
case studies of receiving credit for software contributions; and
awards and recognition that encourage sustainable software

\item Software publishing:
journals and alternative venues for publishing software; and
review processes for published software

\item Software discoverability/reuse:
proposals and case studies

\item Reproducibility and testing:
reproducibility in conferences and journals; and
best practices for code testing and code review

\end{itemize}

Submissions to WSSSPE4 comprised
19 lightning talks,
4 idea papers,
3 position paper,
5 experience papers,
and
3 demos.
Most of the submissions were accepted, since the goal of WSSSPE is always to
take in as many relevant inputs as possible, and to encourage their authors to
participate in sharing and implementing their ideas.
% Specifically,
% 19 lightning talks,
% 4 idea papers,
% 3 position papers,
% 2 experience papers,
% and
% 3 demos
% were accepted, as mentioned in Section~\ref{sec:presentations}.
(Two of the submitted lightning talks and one submitted experience paper were rejected, while two more submitted experience papers
were accepted as lightning talks.)
The papers and lightning talks have been published as a volume in the CEUR Workshop Proceedings~\cite{WSSSPE4-proceedings}.

%%%%%%%%%%%%%%%%%%%%%%%%%%%%%%%%%%%%%%%%%%%%%%%%%%%%%%%%%%%%
\section{Mission and vision}\label{sec:mission}
%%%%%%%%%%%%%%%%%%%%%%%%%%%%%%%%%%%%%%%%%%%%%%%%%%%%%%%%%%%%

%\note{by Lorraine and Dan}

Going into WSSSPE4, WSSSPE had no formal mission or vision statement.
The organizers developed a strawhorse, which was presented to the participants early in the meeting.
The presentation included guidelines as well as examples from other similar communities.
The guidelines were that in general, an organization's mission should stand the test of time and state what the organization does; its vision should imagine what the world would look like if the organization is successful; and from which, focus areas could be used to establish the scope of the organization along with its goals.

The participants were 
%given time after the presentation to write down their  
invited to comment,
% Seven people volunteered to work on 
and 
%revising 
the mission and vision statements, were revised 
%incorporating these comments and presenting back to the community.
based on this feedback.
% the committee redrafted the mission and vision.
The committee added focus areas to clarify the organization's breadth and to keep the mission and vision simple and long-lasting.
Comments after the close of the meeting were also incorporated into the statements.
A final draft was put on GitHub and advertised
via the WSSSPE mailing list, Facebook group, and Twitter.
After two weeks without suggested changes, the final statements, 
%as shown below, 
%were added
to the WSSSPE web page (\url{http://wssspe.researchcomputing.org.uk/about-wssspe/}).

% {\bf Mission.}
% WSSSPE is an international community-driven organization that promotes sustainable research software by addressing challenges related to the full lifecycle of research software through shared learning and community action.

% {\bf Vision.}
% We envision a world where research software is accessible, robust, sustained, and recognized as a scholarly research product critical to the advancement of knowledge, learning, and discovery.

% {\bf Focus areas.}
% WSSSPE promotes sustainable research software by positively impacting:
% \begin{itemize}
% \item {\bf Principles and Best Practices}. Promoting best practices in sustainable software
% \item {\bf Careers}. Developing and supporting career paths in research software development and engineering
% \item {\bf Learning}. Engaging in activities to promote peer learning and interaction
% \item {\bf Credit}. Ensuring recognition of research software as an intellectual contribution equal to other research products
% \end{itemize}

% \textbf{Definitions:}
% \emph{Sustainable software} has the capacity to endure such that it will continue to
% be available in the future, on new platforms, meeting new needs.
% The \emph{research software lifecycle} includes:
% acquiring and assembling resources (including funding and people) into teams and communities,
% developing software,
% using software,
% recognizing contributions to and of software,
% and
% maintaining software.


%%%%%%%%%%%%%%%%%%%%%%%%%%%%%%%%%%%%%%%%%%%%%%%%%%%%%%%%%%%%
\section{Presentations}\label{sec:presentations}
%%%%%%%%%%%%%%%%%%%%%%%%%%%%%%%%%%%%%%%%%%%%%%%%%%%%%%%%%%%%

%\note{from Patricia}

The keynote was given by Patricia Lago and entitled ``The legacy of unsustainable software''.
%
Sustainability is broadly associated with natural ecologic systems. When we translate the notion of sustainability to software solutions, however, we often confuse {\em impact in a certain point in time} with {\em balanced and durable effects}. In addition, software sustainability adds a fourth dimension to environmental, social and economic aspects: technical sustainability, and hence higher complexity~\cite{Lago2015}.  Lago's keynote discussed the challenges (and some related ongoing research) of combining technical and environmental sustainability, providing a complementary angle to the workshop discussions.

%%%%%%%%%%%%%%%%%%%%%%%%%%%%%%%%%%%%%%%%%%%%%%%%%%%%%%%%%%%%
%\section{Papers and demos} \label{sec:papers}
%%%%%%%%%%%%%%%%%%%%%%%%%%%%%%%%%%%%%%%%%%%%%%%%%%%%%%%%%%%%

%\note{Sandra Gesing will write this}

WSSSPE4 included the presentation of 12 10-minute talks (based on four idea papers, three position papers,
two experience papers, and three demos) that addressed a wide range of topics around
sustainability for software in science. These talks covered three main areas.

% The following four papers belong to the first area: the academic environment.
%For talks with multiple authors, an
% \textsuperscript{\textasteriskcentered} indicates the author who presented the talk.
Four papers, 
% \begin{itemize}
% \item Michael Heroux: \emph{Idea paper: Sustainable \& Productive:
% Improving Incentives for Quality Software}
~\cite{Heroux:2016ws}, 
% \item Gabrielle Allen: \emph{Idea Paper: Establishing a Professional
% Society for Research Software}
~\cite{GAllen:2016ws}, 
% \item Olivier Philippe\textsuperscript{\textasteriskcentered}, Simon Hettrick, and Neil Chue Hong:
% \emph{Experience Paper: Preliminary analysis of a survey of UK Research
% Software Engineers}
~\cite{Philippe:2016ws}, and 
% \item Christopher Gwilliams: \emph{Demo: Using industrial engagement to
% create and develop research ties within academia}
~\cite{Gwilliams:2016ws} belong to the first area: the academic environment.
They address diverse aspects ranging from incentives for quality software to
advocating a professional society for research software to roles and degrees for
research software engineers.
%\end{itemize}
The second main area, concerned with characteristics and needs of
research software, includes five papers:
~\cite{Dubey1:2016ws},  ~\cite{ChueHong:2016ws},
~\cite{Dubey2:2016ws}, ~\cite{Queiroz:2016ws}, and ~\cite{Childers:2016ws}.
% \begin{itemize}
% \item Anshu Dubey\textsuperscript{\textasteriskcentered} and Katherine Riley: \emph{Experience Paper: Software
% Engineering and Community Codes Track in ATPESC}
% \item Neil Chue Hong: \emph{Position Paper: Why do we need to compare
% research software, and how should we do it?}
% \item Anshu Dubey\textsuperscript{\textasteriskcentered} and Lois Curfman McInnes: \emph{Idea Paper: The Lifecycle
% of Software for Scientific Simulation}
% \item Francisco Queiroz\textsuperscript{\textasteriskcentered} and Rejane Spitz: \emph{Position Paper:
% Collaborative Gamification Design for Scientific Software}
% \item Bruce Childers\textsuperscript{\textasteriskcentered}, Jack Davidson, Wayne Graves, Bernard Rous, and
% David Wilkinson: \emph{Position Paper: Active Curation of Artifacts is Changing
% the Way Digital Libraries will Operate}
% \end{itemize}
%\noindent 
Three contributions, ~\cite{Ganguly:2016ws}, ~\cite{Shende:2016ws},
and ~\cite{Sallai:2016ws} , focused on the third main area: elucidating lessons learned
from or visions for use cases of scientific software and communities working
with and\slash or on a software package. There were 19 lightning talks presented at WSSSPE4, (in order of
presentation): ~\cite{Sufi:2016ws}, ~\cite{Gesing:2016ws},
~\cite{Ram:2016ws}, ~\cite{Loffler:2016ws}, ~\cite{Katz:2016ws},
~\cite{Idaszak:2016ws}, ~\cite{Hwang:2016ws}, ~\cite{Goble:2016ws},
~\cite{Druskat:2016ws}, ~\cite{Contrastin:2016ws},
~\cite{AAllen:2016ws}, ~\cite{ChueHong:2016wsb},
~\cite{vanHage:2016ws}, ~\cite{GAllen:2016wsb}, ~\cite{Seidel:2016ws}, 
  ~\cite{Emsley:2016ws}, ~\cite{Dongarra:2016ws},
  ~\cite{Bauer:2016ws},  and ~\cite{Aldabjan:2016ws}.


% \begin{itemize}
% \item Debashis Ganguly, William C.\ Garrison III, David Wilkinson,
% Bruce R.\ Childers, Adam J.\ Lee, and Daniel Mosse\textsuperscript{\textasteriskcentered}: \emph{Demo: Composing,
% Reproducing, and Sharing Simulations}
% \item Sameer Shende\textsuperscript{\textasteriskcentered} and Allen Malony: \emph{Demo: Using TAU for
% Performance Evaluation of Scientific Software}
% \item Janos Sallai\textsuperscript{\textasteriskcentered}, Christopher Iacovella, Christoph Klein, and Tengyu Ma:
% \emph{Idea Paper: Development of a Software Framework for Formalizing Forcefield
% Atom-Typing for Molecular Simulation}
% \end{itemize}



%%%%%%%%%%%%%%%%%%%%%%%%%%%%%%%%%%%%%%%%%%%%%%%%%%%%%%%%%%%%
%\section{Lightning talks} \label{sec:lightning}
%%%%%%%%%%%%%%%%%%%%%%%%%%%%%%%%%%%%%%%%%%%%%%%%%%%%%%%%%%%%
\begin{comment}
\note{
\href{http://wssspe.researchcomputing.org.uk/wssspe4/agenda/}{Slides.}}
\end{comment}

%
%For talks with multiple authors, an \textsuperscript{\textasteriskcentered} again indicates the author who presented the talk.

% \begin{itemize}%[itemsep=1ex]
%     \item Shoaib Sufi: \emph{The Software Sustainability Institute Fellowship
%     programme; supporting the social side of research software}

%     \item Sandra Gesing\textsuperscript{\textasteriskcentered}, Maytal Dahan, Linda B.~Hayden, Katherine Lawrence,
%     Suresh Marru, Marlon Pierce, Nancy Wilkins-Diehr, and Michael Zentner:
%     \emph{The Science Gateways Community Institute - Supporting Communities to
%     Achieve Sustainability for Their Science Gateways}

%     \item Karthik Ram\textsuperscript{\textasteriskcentered}, Noam Ross,
%     and Scott Chamberlain: \emph{A model for
%     peer review and onboarding research software}
%     \item Frank L\"{o}ffler and Steven Brandt\textsuperscript{\textasteriskcentered}:
%     \emph{A vision of computing in 10+ years}
%     \item Daniel S.~Katz, Kyle E.~Niemeyer\textsuperscript{\textasteriskcentered},
%     Arfon M.~Smith, and the FORCE11 Software Citation Working Group:
%     \emph{Software Citation: Process, Principles, and
%       Implementation}

%     \item Ray Idaszak\textsuperscript{\textasteriskcentered},
%     David G.~Tarboton, Hong Yi, Michael Stealey,
%     Pabitra Dash, Alva Couch, Daniel P.~Ames, Jeffery S.~Horsburgh, Tony Castronova,
%     Jon Goodall, Mohamed Morsy, Venkatesh Merwade, Mauriel Ramirez, Tian Gan,
%     Drew (Zhiyu) Li, Jeff Sadler, Shawn Crawley, Zhaokun Xue, Lan Zhao, Carol Song,
%     Christina Bandaragoda: \emph{HydroShare - A Case Study in Software Engineering
%     Best Practices and Culture Change for Developing Sustainable Community
%     Software}

%     \item Lorraine Hwang\textsuperscript{\textasteriskcentered},
%     Wolfgang Bangerth, Timo Heister, and Louise Kellogg:
%     \emph{ASPECT: Hackathons as an Example of Sustaining an Open Source
%     Community}

%     \item Carole Goble: \emph{A Simple Profiling Framework for Software
%     User-Producer Reciprocity Review}
%     \item Stephan Druskat: \emph{A proposal for the measurement and
%     documentation of research software sustainability in interactive metadata
%     repositories}

%     \item Mistral Contrastin, Matthew Danish,
%     Dominic Orchard\textsuperscript{\textasteriskcentered}, and
%     Andrew Rice: \emph{Supporting software sustainability with
%     lightweight specifications}

%     \item Alice Allen, Cecilia Aragon, Christoph Becker, Jeffrey Carver,
%     Andrei Chi\c{s}, Benoit Combemale, Mike Croucher, Kevin Crowston, Daniel Garijo,
%     Ashish Gehani, Carole Goble\textsuperscript{\textasteriskcentered},
%     Robert Haines, Robert Hirschfeld, James Howison,
%     Kathryn Huff, Caroline Jay, Daniel S.~Katz, Claude Kirchner, Kateryna Kuksenok,
%     Ralf L\"{a}mmel, Oscar Nierstrasz, Matt Turk, Rob van Nieuwpoort, Matthew Vaughn,
%     and Jurgen Vinju: \emph{``I solemnly pledge'': A Manifesto for Personal
%     Responsibility in the Engineering of Academic
%     Software}

%     \item Neil Chue Hong\textsuperscript{\textasteriskcentered}:
%     \textit{Making it easier to understand research software
%       impact}

%     \item Willem Robert van Hage\textsuperscript{\textasteriskcentered},
%     Jason Maassen and Rob van Nieuwpoort: \textit{Software Impact Measurement at the
%     Netherlands eScience Center}

%     \item Gabrielle Allen, Emily Chen\textsuperscript{\textasteriskcentered},
%     Ray Idaszak, and Daniel S.\ Katz: \textit{Report on Software Metrics for
%     Research Software}

%     \item Eric L.~Seidel\textsuperscript{\textasteriskcentered}
%     and Gabrielle Allen: \textit{Bringing Techniques from Software Engineering
%     into Scientific Software}

%     \item Iain Emsley\textsuperscript{\textasteriskcentered} and David
%     De Roure: \textit{Sustaining the social: Connecting the lives of Drupal
%     community}

%     \item Jack Dongarra, Sven Hammarling, Nicholas J.\ Higham, Samuel
%     D.\ Relton\textsuperscript{\textasteriskcentered}, Pedro Valero-Lara,
%     and Mawussi Zounon: \textit{Creating a Standardised Set of Batched
%     BLAS Routines}

%     \item Lukas Breitwieser\textsuperscript{\textasteriskcentered},
%     Roman Bauer, Alberto Di Meglio, Leonard Johard,
%     Marcus Kaiser, Marco Manca, Manuel Mazzara, Fons Rademakers,
%     Max Talanov: \textit{The BioDynaMo Project}

%     \item Aseel Aldabjan, Robert Haines, and
%     Caroline Jay\textsuperscript{\textasteriskcentered}: \textit{How should we
%     measure the relationship between code quality and software
%     sustainability?}

% \end{itemize}

%%%%%%%%%%%%%%%%%%%%%%%%%%%%%%%%%%%%%%%%%%%%%%%%%%%%%%%%%%%%
%\section{Best practices for scientific software panel discussion} \label{sec:panel}
%%%%%%%%%%%%%%%%%%%%%%%%%%%%%%%%%%%%%%%%%%%%%%%%%%%%%%%%%%%%

%\note{from Simon}

The final WSSSPE4 presentation was a panel %To investigate best practices for scientific software we brought together a panel
of five experts with different perspectives on ``Best Practices for Scientific Software.''
These were: Alice Allen, Editor, Astrophysics Source Code Library, who brought an understanding of the difficulties of organizing a community and curating their software;
Mike Croucher from the University of Sheffield and Rob Haines from the University of Manchester who are both Research Software Engineers with decades of experience of writing code for researchers;
Patricia Lago from Vrije Universiteit Amsterdam, who presented a keynote at the workshop  and brought a fresh perspective on software sustainability from the point of view of its impact on society and business; and
Karthik Ram from the Berkeley Institute for Data Science, who brought his perspective on the practicalities of using scientific software to conduct his research as a data scientist.

%%%%%%%%%%%%%%%%%%%%%%%%%%%%%%%%%%%%%%%%%%%%%%%%%%%%%%%%%%%%
\section{Working groups} \label{sec:WGs}
%%%%%%%%%%%%%%%%%%%%%%%%%%%%%%%%%%%%%%%%%%%%%%%%%%%%%%%%%%%%

After most of the lightning talks and other presentations, WSSSPE used three areas of a large room to let
attendees use sticky flip charts and sticky notes on the walls to suggest a vision on any aspect of the work, a gap or challenge, or a project idea.
Next, attendees organically formed groups around the flip charts, and 12 groups emerged.
Summaries of each group's activities follow; for more details
about the group discussions, see \cite{WSSSPE4-report}.



\subsection{Verifying best practices \& metrics for sustainable research software}
\label{sec:best-practices-sustainable}
%%%%%%%%%%%%%%%%%%%%%%%%%%%%%%%%%%%%%%%%%%%%%%%%%%%%%%%%%%%%

%\note{Ray to write this}

Many open source projects for research software document their best practices that
contribute to the sustainability of the software.  Many of these projects also
document software metrics they use to define their research software project's success.
%
This group will take the outputs of the WSSSPE efforts to identify best practices for creation of sustainable software for science/academia, and also the outputs of WSSSPE efforts to identify metrics for sustainable software for science/academia and cross-reference these with current open source research software that successfully uses modern software engineering.  This will allow the group to identify gaps on both sides.  This approach will also allow the group to hypothesize how successful open source projects can be further improved, verify that recommended approaches for software engineering for science/academia are sufficient and valid, and that metrics for software engineering for science/academia are relevant and useful.  The group has a specific objective of getting a good cross-sampling of disparate software including community model codes, community cyberinfrastructure, community analytic tools, etc.

Each group member will volunteer to take on one to two projects, ideally ones in which they are already interested.  This effort will evaluate an initial 5--10 projects.  Once the group documents the workflow by which these evaluations are performed, other people can choose to follow this workflow and contribute their own evaluations of additional software.

Anyone can join this group or obtain more information about it, by sending an email to the verifying best practices \& metrics for sustainable research software group: <wssspe4-verify-best-practices@googlegroups.com>.


%%%%%%%%%%%%%%%%%%%%%%%%%%%%%%%%%%%%%%%%%%%%%%%%%%%%%%%%%%%%
\subsection{Software Sustainability Alliance}
\label{sec:alliance}
%%%%%%%%%%%%%%%%%%%%%%%%%%%%%%%%%%%%%%%%%%%%%%%%%%%%%%%%%%%%

%\note{Neil and Jean wrote this}

The Software Sustainability Alliance working group aims to establish an alliance between organizations interested in improving the sustainability of research software. Such an alliance would ease the collaboration between member organizations to improve the sharing of expertise, resources and best practices.
These organizations are funded groups or teams that aim to advance research software sustainability beyond their local university or community. More specifically, this working group aims to define the scope of this alliance and provide clear distinction with WSSSPE. The group also seeks to understand the incentives for members of this alliance and identify its key activities.
It is currently seeking feedback on potential member organizations, as well as the aims and scope of this alliance.

Currently, point-to-point collaboration exists between organizations, but this inadvertently results in competition or redundancy within the sustainable software community. An alliance of software sustainability organizations would ease inter-organization collaboration and the promotion of software sustainability. This alliance would also improve the pooling of competencies and the sharing of expertise. Furthermore, with the international scope of this alliance, it could support an organization who wants to hold an event in a country where another organization exists.

If interested, one can visit \url{http://softwaresustainability.org/} or email Neil Chue Hong (<N.ChueHong@software.ac.uk>) and Jean Salac (<jeansalac@virginia.edu>)

%%%%%%%%%%%%%%%%%%%%%%%%%%%%%%%%%%%%%%%%%%%%%%%%%%%%%%%%%%%%
\subsection{Scientific Software Prototyping Infrastructure (S2PI)}
\label{sec:prototyping}
%%%%%%%%%%%%%%%%%%%%%%%%%%%%%%%%%%%%%%%%%%%%%%%%%%%%%%%%%%%%

%from Santiago
There is a productivity bottleneck---yet to be solved---in HPC from the human
perspective, first identified in the 1980s~\cite{barstow1982automatic}. Of all
domain-specific scientists, only a percentage of those use simulations due to
perceptions mostly related to the difficulty of using and developing those tools.
XSEDE and other resources have become a gateway for successfully increasing the
basis of HPC scientific users by facilitating their access to infrastructure and
tools~\cite{towns2014xsede}, but the development challenge remains.
Of all scientists who use simulations, only a small percentage know how
to develop prototypes or full applications that can be later scaled by computer
scientists and engineers. Writing new parallel applications is unusual for
domain scientists and thus has created a complex dependency on specialized
scientific programmers, who are scarce~\cite{post2005computational}. However,
the capacity to quickly envision and prototype applications by domain
experts---later to be fully implemented by scientific programmers---is critical for
acquiring maturity and proficiency in the grand challenges that Exascale
attempts to solve, and moreover for allowing scientists to develop their own
ideas quickly and productively~\cite{vinter2015prototyping}. The latter, in
addition, would increase service utilization time in HPC systems, a critical
measure of efficiency. Considering the existing code base in scientific
computing, software infrastructures that allow a clean transition from legacy
systems to new Exascale platforms while preserving flexible prototyping
towards the future are central~\cite{hwu2015transitioning}.

This group deems the ability to rapidly construct software artifacts that can be
trusted in terms of the methods from their design up, easily discarded when
wrong and extended when right at low human and computational cost to be one
aspect of scientific software sustainability. In essence,
a motto for scientific software prototyping is \textit{fail hard, fail fast}.
%
The group's objective is development of a prototype tool that allows domain scientists to generate
simulation prototypes through a simple, yet expressive declarative
strongly-typed programming language close to equational expressions. The result
of that specification is both an executable artifact as well as a specification
for scientific programmers to later flesh out completely and adapt to particular
infrastructures. This tool will start in the domain of computational physics,
and (probably) extend to other domains, and the code will be open source.
%
The underlying hypothesis of this project is that a strongly typed, declarative
problem-specification language with adequate constructs, to be implemented
through stencils and algorithmic skeletons tied to numerical methods, captures a
 subset of useful and interesting problems in physics and other domains.
%
The group plans to develop a functional (meta-)prototype before WSSSPE5, and
if useful, then develop a larger software infrastructure tied to
particular cyberinfrastructure resources.

To join this project, one can visit the GitHub repository at
\url{https://github.com/nunezco2/S2PI} or join the WSSSPE Slack channel
\texttt{\#wg-sci-soft-proto}.


%%%%%%%%%%%%%%%%%%%%%%%%%%%%%%%%%%%%%%%%%%%%%%%%%%%%%%%%%%%%
\subsection{CodeMeta}
\label{sec:CodeMeta}
%%%%%%%%%%%%%%%%%%%%%%%%%%%%%%%%%%%%%%%%%%%%%%%%%%%%%%%%%%%%

%\note{Alice to write this}

Research software is often not shared; that which is shared may not have much metadata associated with it, and that which does exist often does not travel further than the website on which the software resides. The CodeMeta project\footnote{\url{http://codemeta.github.io/}} wants to incentivize software developers to release their software, encourage the development of metadata for it, enable credit assignment and citation of research software, increase its discoverability, more easily track dependencies, and enable reuse of software metadata, all goals that WSSSPE attendees have great interest in supporting.  CodeMeta seeks in part to create a ``Rosetta Stone'' for software metadata to facilitate retaining such metadata between repositories, services, registries, indexers, publishers, citation managers, and other entities that create, ingest, use, and/or store metadata about software. The project also wants to establish a JSON-LD schema as a tool for making metadata machine-readable~\cite{CodeMeta_schema}.

The primary objective of this working group, which included people who had previously been working on the CodeMeta project, is to find ways to help the project come to fruition.
%
The working group plans to engage with those already working on CodeMeta, to examine the existing CodeMeta crosswalk table to see what improvements and additions might be made, and to determine how to engage the community and provide ongoing social engagement and structure. Further, the group wants to assist in the implementation of the specifications and, by providing an outsider's view, contribute suggestions for more understandable project documentation. Finally, the group seeks a better way or ways to present the crosswalk table to make it more easily understood and consumable by research software communities.
%
While at WSSSPE4, the group greatly expanded the CodeMeta project README file to include a description of the project geared to those with little or no prior knowledge of the project, a list of contributors, information on how one can get involved, a brief project history and who is managing the project, and links to additional information.

Though a Google group mailing list has been established for the working group, the easiest way to engage with the CodeMeta project is through its Github repository\footnote{\url{https://github.com/codemeta/codemeta}}.


%%%%%%%%%%%%%%%%%%%%%%%%%%%%%%%%%%%%%%%%%%%%%%%%%%%%%%%%%%%%
\subsection{White paper on developing sustainable software}
\label{sec:best-practices-developing}
%%%%%%%%%%%%%%%%%%%%%%%%%%%%%%%%%%%%%%%%%%%%%%%%%%%%%%%%%%%%

%\note{Sandra to write this}

Many diverse aspects and dimensions (such as economic, technical, environmental, and social) of developing sustainable software can be investigated,
This group aims to write white papers
that will focus on scientific environments and their implications, targeted at developers
and project managers of scientific software. Given the complexity of this field, it is important to select
a subset of sustainability aspects for the white papers. The idea is to create a series of papers instead of trying
to tackle all important topics in one paper.
%
While there are already a few papers available on best practices and sustainability of scientific software in general, the group's goal is to create a series of papers that lead to consensus in the community, tackle many diverse aspects, and stay up-to-date with new trends.
%
For the first white paper, the group aims to set the
stage with successful use cases and an analysis of why they have been successful.
Further topics will include community-related practices, government and management, funding,
metrics, tools, and usability.
The resulting draft white paper will be distributed via the WSSSPE email list. After collecting feedback on it, the group's plans are to attract a wider community to contribute to an extended journal paper version and to investigate whether more people would like to be involved in the white paper series.

The GitHub repository for the white paper can be found at \url{https://github.com/WSSSPE/WG-Best-Practices}. For more information and requests, join the \texttt{\#wg-best-practices} channel at WSSSPE's Slack team.



%%%%%%%%%%%%%%%%%%%%%%%%%%%%%%%%%%%%%%%%%%%%%%%%%%%%%%%%%%%%
\subsection{Social science for scientific software}
\label{sec:social}
%%%%%%%%%%%%%%%%%%%%%%%%%%%%%%%%%%%%%%%%%%%%%%%%%%%%%%%%%%%%

%\note{from Stuart}

%Introduction to group here, including the overall objective of work in this area.



There is more and more academic research being done on topics related to software sustainability, including work on software engineering practices and management of open source projects. However, academic research in general is often siloed for many reasons, and work on topics relevant to software sustainability is no exception. There are many academic studies and projects which may be relevant for practitioners working in this area, and many projects and initiatives by practitioners that may be relevant for social scientists. However, there is a gap between these two in research and practice.
%
This working group is motivated by the goal of building better connections between academic researchers who are studying topics in or relating to software sustainability with practitioners, managers, and administrators who are working in the area of software sustainability.
%
 In any domain, bringing research and practice closer together a mutually beneficial goal, but also has its challenges.
The group aims to give social scientists and practitioners working on scientific software, software sustainability, and open source and a better understanding of each other's work, as well as help them connect and coordinate on specific projects of mutual interest.
%
Furthermore, the group also recognizes that academic social science and research software engineering are not monolithic, and there is a need to connect people who care about research-driven best practices inside of these two domains with each other as well.

The working group plans to briefly survey existing literature to get a better sense of the academic research landscape, facilitate some initial dialog between academic researchers and practitioners at WSSSPE4, then identify needed actions that would be mutually beneficial to researchers and practitioners. This plan resulted in the creation of the following SMART steps listed in the next section.


To join the group, one can join its mailing list at \url{https://groups.google.com/forum/#!forum/researchsoftwarestudies}


%%%%%%%%%%%%%%%%%%%%%%%%%%%%%%%%%%%%%%%%%%%%%%%%%%%%%%%%%%%%
\subsection{Software best practices for undergraduates}
\label{sec:best-practices-undergrads}
%%%%%%%%%%%%%%%%%%%%%%%%%%%%%%%%%%%%%%%%%%%%%%%%%%%%%%%%%%%%

%\note{Jonah Miller to write this}

This working group was motivated by the perceived prevalence of
so-called ``hidden code'' in scientific communities: code written by
individual researchers in an unsustainable way that is never shared
with the larger community. Participants recalled their own experience
working with colleagues who write hidden code or inheriting
unsustainable hidden code from collaborators.
%
The question is, then, how to prevent researchers from writing hidden
code? The participants hypothesized that the best strategy is to catch
researchers while they are still in training and teach them software
best-practices. Therefore, this working group's goal
is to develop courses on software best practices aimed at
undergraduate students studying domain science. The program might be
similar to a Software Carpentry or Data Carpentry workshop but
focused for domain scientists.

While implementation of a course aimed at each domain science is a long term
goal, a short-term, achievable goal, is to develop a
curriculum for a course aimed at a single domain science. Since the
participants in the working group have expertise in physics, this is a
natural target.
%
The development of a successful curriculum relies on the expertise of
software engineers to describe the best practices, domain experts to
describe model problems and work-flow, and instructors to formulate
the pedagogy. Ideally these people will be brought together for a
short workshop or hackathon with the goal of drafting the
curriculum. The
hackathon will be implemented and a draft of the curriculum will be
written within the next year.
Later, organizational partners will be required to
actually implement the program.


Anyone interested in contributing should get in touch via one of the
following methods:

\begin{itemize}
\item Join the Google Group/mailing list
  \cite{WSSSPEUndergradGoogleGroup}.
\item Join the WSSSPE Slack (\S\ref{sec:slack}) and the
  \texttt{\#wg-undergraduatecourse} channel.
\item Ask to join the WSSSPE4-undergraduate-course organization
  \cite{WSSSPEUndergradGithub}.
\end{itemize}



%%%%%%%%%%%%%%%%%%%%%%%%%%%%%%%%%%%%%%%%%%%%%%%%%%%%%%%%%%%%
\subsection{Meaningful metrics for sustainable software}
\label{sec:metrics}
%%%%%%%%%%%%%%%%%%%%%%%%%%%%%%%%%%%%%%%%%%%%%%%%%%%%%%%%%%%%

%\note{content from Emily}

This group aims to increase the visibility of the quality of scientific software, facilitate the reusability of scientific software, and promote the best software practices by standardizing metrics via interviews with scientific software developers. This working group believes improving the current software metrics system will increase software sustainability. Currently, there are inefficiencies regarding software duplication, sustainability, and selection, as well as others, within the scientific software community. In order to address these inefficiencies, the group aims to create a goal-oriented method to collecting productive metrics by focusing on the developer side of software.

In the context of this aim, the group hopes to find an efficient solution for streamlining the process of collecting and utilizing metrics to benefit software sustainability, as well as minimize the current inefficiencies within the scientific software community. This will improve software evaluation and comparison, thus reducing the effort spent on seeking scientific software.
%
The group plans on surveying and interviewing scientific software developers to form metrics from the goals they have for their software, then to publicize the results.

For more information, one can contact Emily Chen at <echen35@illinois.edu>.



%%%%%%%%%%%%%%%%%%%%%%%%%%%%%%%%%%%%%%%%%%%%%%%%%%%%%%%%%%%%
\subsection{Coordinating access to continuous integration (CI) for research software}
\label{sec:access}
%%%%%%%%%%%%%%%%%%%%%%%%%%%%%%%%%%%%%%%%%%%%%%%%%%%%%%%%%%%%

%\note{%Mark Abraham to write this?  Needs confirmation or another volunteer -
%Dan starting this for now - may poll other members of group to make progress}

Each developer of software with uncommon needs (hardware, software, libraries, data sets), non-public code, or tests that exceed time limits for free plans must acquire, setup, and maintain their own continuous integration systems because their needs make them ineligible for popular free services such as Travis CI.  For example, software groups that develop BIOUNO, CI4SI, or GROMACS have done this.
(Note that Debian provides a testing infrastructure\footnote{\url{https://ci.debian.net/doc/}} that is mature and supports hardware and other requirement specifications.)

This group is interested in reducing the burden of different projects having to build and maintain their own continuous integration systems (when publicly available CI are not a fit), by coordinating and sharing this burden across multiple projects.
%
The scope of the group's interest is any type of testing, though interactive access for troubleshooting would be of particular interest beyond just automated testing.

However, there are a lot of open issues:
\begin{itemize}
\item Some similar work has already been done.  How can this group apply and/or learn from that work, rather than reinventing the wheel?
\item How to ensure this will work across disciplines?
\item Meaningful CI for large projects may need hundreds of CPU hours per day
\end{itemize}

Some possible goals include:
\begin{itemize}
\item Acquire additional hardware such as GPUs, Xeon PHI, FPGAs and add/share them to e.g. Debian's testing infrastructure or to a shared Jenkins-based infrastructure
\item Extending Debian's scope to include published but not mature software
\item Implementing the same interface but with specialized hardware or available to non-public codes.
\item \url{https://reproducible-builds.org} but for CI
\end{itemize}

For more information, see \url{https://groups.google.com/forum/#!forum/continuous-integration-for-research-software}




%%%%%%%%%%%%%%%%%%%%%%%%%%%%%%%%%%%%%%%%%%%%%%%%%%%%%%%%%%%%
\subsection{Software engineering processes tailored for research software}
\label{sec:soft-eng}
%%%%%%%%%%%%%%%%%%%%%%%%%%%%%%%%%%%%%%%%%%%%%%%%%%%%%%%%%%%%

%\note{Anshu to write this}

This working group is concerned with identifying processes that are not adequately
covered by general software engineering, starting with verification and testing.
Computational science and engineering applications have many moving
parts that need to interoperate with one another. The accuracy and
reliability of results produced by scientific software depends not
only on the individual components behaving correctly, but also on the
validity of their interactions.
As scientific understanding grows,
the corresponding computational software models are refined, leading
to more complex codes. Increasing complexity makes them more prone to
defects, not only in individual code units, but also in interaction
among units. Therefore, a strong verification process combined with a
rigorous testing regime plays a critical role in the prevention of
generating incorrect scientific results. However, most science teams
struggle to find a good solution for themselves.

One cause is lack of exposure to the practices. While
good developers will test their code to
verify that it operates as expected, they may not appreciate
that without regular testing, defects can be introduced
inadvertently.
%
An even bigger challenge is that those who understand
the importance of regular testing do not often find much help from
software engineering literature. There is a significant gap between
the testing gospel and its applicability to computational science. This
gap leads to frustration and abandonment of the good with the bad. Some
relevant literature exists, in particular experiences from
practitioners in computational science who developed their own
solutions. However, this literature is scattered among many different
forums, and can be challenging to find. This working group aims to
address this gap by curating the existing content and contributing
content where none exists.

This working group will (1) conduct a literature survey to gauge the extent
of awareness of the issue in general, (2) generate content useful for
the community where needed, and (3) curate the collected and added
content for the use of the community.

Readers interested in getting more information should use the working group'
channel in the WSSSPE Slack (\S\ref{sec:slack}), called
\texttt{\#wg-testing-in-science}.
Additionally, a git
repository (\url{https://github.com/WSSSPE/WG-Best-Practices.git}) exists for contributing content and reference to, and
curation of the existing literature on this topic.




%%%%%%%%%%%%%%%%%%%%%%%%%%%%%%%%%%%%%%%%%%%%%%%%%%%%%%%%%%%%
\subsection{Open research index}
\label{sec:open-research-index}
%%%%%%%%%%%%%%%%%%%%%%%%%%%%%%%%%%%%%%%%%%%%%%%%%%%%%%%%%%%%

%\note{Dan to write this}

The aim of this group is to investigate the building of an index of research products in an open sustainable manner.  Its goal is not to eliminate commercial products, but to build on what is there and provide data and services that are missing.
%
The Open Research Index should take in all research products (papers, software, datasets, workflows, etc.) from their publishers and recorders (journals, societies, domain repositories, government [open access] repositories, preprint servers, general repositories [e.g., figshare, zenodo]) and other services (CrossRef, ORCID).
Each product should list authors and citations and allow people to search the resulting network.
Users should also be able to interact with their own record and edit it, like Google Scholar allows.

The working group's plans are relatively simple to express, though quite complex to undertake.  They are to first determine a plan to build an open research index that allows various stakeholders to satisfy their needs, then to determine if the plan is feasible.
At the time of the meeting, and today as well, it is unclear who has the time and energy to pursue this idea. Thus, a leader needs to be identified.
With a leader and feasible plan, the group would then obtain resources, and then build the index.
%
Because Google and others provide some similar services today, though these services (and the underlying data) could be removed at any time and the community cannot build new services, these companies would ideally be involved in the activities of the group.

Alternatively or additionally, the group could build a mailing list for us and discuss further, depending on how receptive others are to this idea.  At WSSSPE4, there seemed to be enough interest to do this, so the group set up a Slack channel within the WSSSPE team (\S\ref{sec:slack}), called \texttt{\#wg-open-research-idx} for its members and any future members.




%%%%%%%%%%%%%%%%%%%%%%%%%%%%%%%%%%%%%%%%%%%%%%%%%%%%%%%%%%%%
\subsection{Letters of evaluation for computational scientists}
\label{sec:letters}
%%%%%%%%%%%%%%%%%%%%%%%%%%%%%%%%%%%%%%%%%%%%%%%%%%%%%%%%%%%%

Scientists working on scientific software are often located in
disciplinary departments, depending on whether their software
originates from the mathematical, physical, chemical, or other
disciplines. As a consequence, they are frequently outside the core
areas of their science, and their contributions are typically to both
the research activity their software enables, as well as on
algorithm and implementation aspects. This presents issues when
letters of evaluation for hiring, tenure, and promotion do not specifically
cover how this is relevant to the discipline.\footnote{The extensive use of such letters, and the problems that are
  associated with it, may be an issue specific to the United States.}

This group believes that the authors of scientific software provide important
services to departments that are no less than strictly disciplinary
research. Consequently, leveling the playing field with more
disciplinary candidates for hires, tenure, and promotion requires that
letter writers be aware of how their letters will be read by
committees. It also requires that committees be aware that such
letters often look different and may provide a different perspective
of how a candidate's achievements should be assessed. For example, in
mathematics a typical candidate would be evaluated on the difficulty
and depth of the statements she may have proven in their papers. On
the other hand, an author of mathematical software would likely be
evaluated based on the impact of her software, or the number of citations of
the publications that describe it. She may also be evaluated by how
widely the software is used \textit{outside} mathematics, a criterion
that is rarely used for more disciplinary mathematicians.

Addressing this problem likely requires that letter writers pay
particular attention to who exactly the audience of such letters is,
and tailor the message by specifically highlighting how the work of a
candidate benefits the department and discipline that is considering a
candidate. On the other hand, committee members also need to pay
attention to the fact that there are areas that are important to the mission of
their department and professional community in which different
standards for evaluation hold.
%
There are essentially two important strategies that build on each
other: (i) raising awareness of the problem beyond just those who are
affected by it (namely, computational scientists working on scientific
software), and (ii) providing letter writers, letter readers, and
evaluating committees with guidance on what criteria are relevant in
assessing scientific software authors.

Concrete guidance is most valuable if it
comes from respected bodies such as professional societies or
established and respected organizations. Getting these to act
will only be possible if the problem is widely acknowledged, and so
the first of the goals above should be the current focus.
%
The group will attempt to address it by organizing sessions at conferences
that address the career problem, as well as writing editorials that
can be published in the magazines of professional societies. These
editorials ought to outline best practices for letter writers that
specifically (i) make clear the contribution of a candidate to their
interdisciplinary area, (ii) the relevance to their home discipline, and
(iii) why writing software is good for the discipline itself.

Ultimately, building a community large enough
to affect change is important; contact Wolfgang Bangerth at
<bangerth@colostate.edu> if you are interested in helping.



%%%%%%%%%%%%%%%%%%%%%%%%%%%%%%%%%%%%%%%%%%%%%%%%%%%%%%%%%%%%
\section{Slack team and channels}\label{sec:slack}
%%%%%%%%%%%%%%%%%%%%%%%%%%%%%%%%%%%%%%%%%%%%%%%%%%%%%%%%%%%%


As part of the discussion at WSSSPE4, we created a Slack team for WSSSPE~\cite{WSSSPESlack}.  To join it, go to \url{https://wssspe.signup.team/}.  A number of the working groups also created channels within the team; see the subsections in \S\ref{sec:WGs} for the names of these channels.


%%%%%%%%%%%%%%%%%%%%%%%%%%%%%%%%%%%%%%%%%%%%%%%%%%%%%%%%%%%%
\section{Attendee survey \label{sec:survey}}
%%%%%%%%%%%%%%%%%%%%%%%%%%%%%%%%%%%%%%%%%%%%%%%%%%%%%%%%%%%%

%\note{from Lorraine}

All participants present on the last day of the meeting, 14 September 2016, were asked to complete the online survey administered through Qualtrics\footnote{\url{https://www.qualtrics.com}.}.
44 responses were recorded, with one respondent indicating that they took the survey twice.
The survey URL was also later distributed to attendees by email and advertised on twitter for attendees who were not there on the last day to have a chance to respond, and one additional response was recorded after the meeting.
Hence, the total number of unique individuals responding to the survey is likely 44.
The survey contained nine multiple choice and open-text answer questions.
Text responses have been alphabetized to preserve anonymity.
The survey had a 100\% completion rate.

In general, the respondents were highly satisfied with the meeting and interested in continuing and participating in WSSSPE activities.
The mix of topics and types of interactions---talks, panels and discussion---was well balanced with several indicating a desire for time for questions and answers following talks as well for discussions in general.
Many respondents indicated that they will remain engaged in WSSSPE working group initiatives.
However, demand for a professional organization encompassing WSSSPE interests was weaker with over half interested in joining and a large percentage willing to consider the idea.
Respondents were grateful for the opportunity to network, explore new collaborations and for travel support to the conference.

The survey results indicate that future WSSSPE conferences should consider the balance of the attendees.
This year's meeting had many first time and early career participants who would have benefited from a review of basic concepts and terminology including what it means for software to be sustainable and the roles of research software engineers.
Additional emphasis and topics to explore would be the inclusion of more case studies, focus on the decision making process in developing and using software, deeper dives into selected topics, tutorials, software training within and outside of STEM, lowering the barriers to implementing best practices in software development, and progress towards executing WSSSPE's vision.

%%%%%%%%%%%%%%%%%%%%%%%%%%%%%%%%%%%%%%%%%%%%%%%%%%%%%%%%%%%%
\section{Conclusions} \label{sec:conclusions}
%%%%%%%%%%%%%%%%%%%%%%%%%%%%%%%%%%%%%%%%%%%%%%%%%%%%%%%%%%%%

In WSSSPE4, we heard about a number of interesting projects and ideas, and used those
ideas to create and form working groups, intended to start at the workshop and then
continue afterwards, to address challenges that arose from the workshop ideas.  This workshop
has reinforced the lesson from WSSSPE3 that it is relatively easy to get motivated people
to attend a meeting and productively spend their time there both doing work and planning
more work, but it is very hard to get that additional work after the meeting to take place.
The main problem seems to be one of time.  Once the attendees have agree to spend
their time at the workshop, they put their energy into doing so productively, but they have not
really committed themselves to anything more than this, so their energy and effort trails
off, as all of their other commitments (particularly, those they are funded to do as part of
their jobs) come back to the fore. Without a process to resolve this concern, the utility of
further multi-day WSSSPE workshops is unclear.

%\todo{discuss Twitter activity}


%%%%%%%%%%%%%%%%%%%%%%%%%%%%%%%%%%%%%%%%%%%%%%%%%%%%%%%%%%%%
\section*{Acknowledgments} \label{sec:acks}
%%%%%%%%%%%%%%%%%%%%%%%%%%%%%%%%%%%%%%%%%%%%%%%%%%%%%%%%%%%%

% NSF grant for WSSSPE4
This material is based upon work supported by the National Science Foundation (ACI-1648293, ACI-1547611),
by the Gordon and Betty Moore Foundation (GBMF\#5620),
and by the Alfred P.~Sloan Foundation (G-2016-7214).

%\todo{feel free to add stuff here}

\newpage
\appendix
%%%%%%%%%%%%%%%%%%%%%%%%%%%%%%%%%%%%%%%%%%%%%%%%%%%%%%%%%%%%
\section{Organization}  \label{sec:orgcom}
%%%%%%%%%%%%%%%%%%%%%%%%%%%%%%%%%%%%%%%%%%%%%%%%%%%%%%%%%%%%
%\todo{Do we want email addresses here?}

WSSSPE4 organizers were:

{\scriptsize
\begin{longtable}{lll}
\input{orgcom}
\end{longtable}
}


The WSSSPE4 program committee comprised:

{\scriptsize
\begin{longtable}{lll}
David Abramson & University of Queensland, Australia\\
Lorena A. Barba & George Washington University, USA\\
Ross Bartlett & Sandia National Laboratories, USA\\
Christoph Becker & University of Toronto, Canada\\
Ewout van den Berg & IBM Watson, USA\\
David Bernholdt & Oak Ridge National Laboratory, USA\\
Stefanie Betz & Karlsruhe Institute of Technology, Germany\\
Coral Calero & Universidad Castilla La Mancha, Spain\\
Ishwar Chandramouli & National Cancer Institute, National Institutes of Health, USA\\
Ruzanna Chitchyan & University of Leicester, UK\\
Karen Cranston & Duke University, USA\\
Ewa Deelman & Information Sciences Institute, University of Southern California, USA\\
Charlie E. Dibsdale & O-Sys, Rolls Royce PLC, UK\\
Anshu Dubey & Argonne National Laboratory, USA\\
Nadia Eghbal & Independent Researcher (via Ford Foundation), USA\\
Peter Elmer & CERN, Switzerland\\
Martin Fenner & DataCite, Germany\\
David Gavaghan & University of Oxford, UK\\
Mike Glass & Sandia National Laboratories, USA\\
Carole Goble & University of Manchester, UK\\
Joshua Greenberg & Alfred P. Sloan Foundation, USA\\
Michael K Griffiths & University of Sheffield, UK\\
Sarah Harris & University of Leeds, UK\\
James Hetherington & University College London, UK\\
Fred J. Hickernell & Illinois Institute of Technology, USA\\
Neil Chue Hong & Software Sustainability Institute, University of Edinburgh, UK\\
Caroline Jay & University of Manchester, UK\\
Rafael C. Jimenez & ELIXER, Cambridge, UK\\
Matthew B. Jones & University of California Santa Barbara, USA\\
Nick Jones & New Zealand eScience Infrastructure (NeSI), NZ\\
Sedef Akinli Kocak & Ryerson University, Canada\\
Jong-Suk Ruth Lee & National Institute of Supercomputing and Networking, KISTI, Korea\\
James Lin & Shanghai Jiao Tong University, China\\
Frank Löffler & Louisiana State University, USA\\
Gregory Madey & University of Notre Dame, USA\\
Ketan Maheshwari & University of Pittsburgh, USA\\
Steven Manos & University of Melbourne, Australia\\
Chris A. Mattmann & NASA JPL; University of Southern California, USA\\
Abigail Cabunoc Mayes & Mozilla Science Lab, USA\\
Robert H. McDonald & Indiana University, USA\\
Lois Curfman McInnes & Argonne National Laboratory, USA\\
Alberto Di Meglio & CERN, Switzerland\\
Chris Mentzel & Gordon and Betty Moore Foundation, USA\\
Peter Murray-Rust & University of Cambridge, UK\\
Christopher R. Myers & Cornell University, USA\\
Jarek Nabrzyski & University of Notre Dame, USA\\
Cameron Neylon & Curtin University, Australia\\
Aleksandra Pawlik & New Zealand eScience Infrastructure (NeSI), NZ\\
Fernando Perez & Lawrence Berkeley National Laboratory; University of California, Berkeley, USA\\
Marian Petre & The Open University, UK\\
Marlon Pierce & Indiana University, USA\\
Andreas Prlic & University of California, San Diego, USA\\
Karthik Ram & University of California, Berkeley, USA\\
Morris Riedel & Juelich Supercomputing Centre, Germany\\
Dave De Roure & Oxford e-Research Centre, University of Oxford, UK\\
Norbert Seyff & University of Zurich, Switzerland\\
Arfon Smith & GitHub Inc, USA\\
Borja Sotomayor & University of Chicago, USA\\
Edgar Spalding & University of Wisconsin, USA\\
Maria Spichkova & RMIT University, Australia\\
Victoria Stodden & University of Illinois Urbana--Champaign, USA\\
Matthew Turk & University of Illinois Urbana--Champaign, USA\\
Nancy Wilkins-Diehr & San Diego Supercomputer Center, University of California, San Diego, USA\\
James Willenbring & Sandia National Laboratories, USA\\
Scott Wilson & Cetis LLP, UK\\
Theresa Windus & Iowa State University and Ames Laboratory, USA\\

\end{longtable}
}


\newpage
\bibliographystyle{vancouver}

\bibliography{wssspe}
\end{document}
