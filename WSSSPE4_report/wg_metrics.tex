%%%%%%%%%%%%%%%%%%%%%%%%%%%%%%%%%%%%%%%%%%%%%%%%%%%%%%%%%%%%
\subsection{Meaningful metrics for sustainable software}
\label{sec:metrics}
%%%%%%%%%%%%%%%%%%%%%%%%%%%%%%%%%%%%%%%%%%%%%%%%%%%%%%%%%%%%

\note{content from Emily}

%Introduction to group here, including the overall objective of work in this area.

Meaningful Metrics for Sustainable Scientific Software aims to increase the visibility on the quality of scientific software, facilitate the reusability of scientific software, and promote the best software practices by standardizing metrics via interviews with scientific software developers. This working group believes improving the current software metrics system will increase software sustainability. Currently, there are inefficiencies regarding software duplication, sustainability, and selection, as well as others, within the scientific software community. In order to address these inefficiencies, Meaningful Metrics for Sustainable Scientific Software aims to create a goal-oriented method to collecting productive metrics by focusing on the developer side of software.

\subsubsection{Participants}

%members of the working group

\begin{itemize}
\item Emily Chen <echen35@illinois.edu>
\item Patricia Lago <pdotlago@gmail.com>
\item Udit Nangia <unangi2@illinois.edu>
\item Tengyu Ma <tengyuma10717@gmail.com>
\item Aseel Aldabjan <a.dabjan@hotmail.com>
\end{itemize}

\subsubsection{Working group objective}

%Specific things the working group wants to accomplish in the context of the larger objective.

In the context of the larger objective, Meaningful Metrics for Sustainable Scientific Software hopes to find an efficient solution for streamlining the process of collecting and utilizing metrics to benefit software sustainability, as well as minimize the current inefficiencies within the scientific software community. Finding meaningful metrics will improve software evaluation and comparison, thus reducing the effort spent on seeking scientific software.

\subsubsection{Gap or challenge}

%What is the gap or challenge being addressed?

The gap being addressed is the lack of a standard for collecting and presenting metrics. This gap delays workflow and creates a multitude of tedious tasks, including searching for the best-fit software, unknowingly duplicating software, and other related busy-work. The need for metric standardization stems from the abundance of scientific, both ?dark? and open source, software and the difficulties of ensuring the software is sustainable. 

\subsubsection{Relevant people and resources}

%What people, groups, or resources are needed.

Relevant people and resources include scientific software developers, researchers that use scientific software, scientific software funding institutions. 

\subsubsection{Plans}

%What tasks will the working group undertake

This working group plans on interviewing scientific software developers to form metrics from the goals they have for their software. 

\subsubsection{SMART steps}

%What are the first SMART steps proposed?

The SMART steps Meaningful Metrics for Scientific Software proposed are:
\begin{enumerate}
\item Identify the population of scientific software developers who are willing to be interviewed
\item Define the interview questions and organize them into categories
\item Interview the participants and map the survey results to goals
\item Convert the goals to feasible, meaningful metrics
\item Analyze the collected metrics
\end{enumerate}

\subsubsection{More information \& joining instructions}

%How could a reader get more information or get more involved?

For more information, please contact Emily Chen at <echen35@illinois.edu>.
