%%%%%%%%%%%%%%%%%%%%%%%%%%%%%%%%%%%%%%%%%%%%%%%%%%%%%%%%%%%%
\subsection{Scientific Software Prototyping Infrastructure (S2PI)}
\label{sec:prototyping}
%%%%%%%%%%%%%%%%%%%%%%%%%%%%%%%%%%%%%%%%%%%%%%%%%%%%%%%%%%%%

%from Santiago
There is a productivity bottleneck---yet to be solved---in HPC from the human
perspective, first identified in the 1980s~\cite{barstow1982automatic}. Of all
domain-specific scientists, only a percentage of those use simulations due to
perceptions mostly related to the difficulty of using and developing those tools.
XSEDE and other resources have become a gateway for successfully increasing the
basis of HPC scientific users by facilitating their access to infrastructure and
tools~\cite{towns2014xsede}, but the development challenge remains unaddressed.
Of all scientists who use simulations, only a small percentage know how
to develop prototypes or full applications that can be later scaled by computer
scientists and engineers. Writing new parallel applications is unusual for
domain scientists and thus has created a complex dependency on specialized
scientific programmers, who are scarce~\cite{post2005computational}. However,
the capacity to quickly envision and prototype applications by domain
experts---later to be fully implemented into by scientific programmers---is critical for
acquiring maturity and proficiency in the grand challenges that Exascale
attempts to solve, and moreover for allowing scientists to develop their own
ideas quickly and productively~\cite{vinter2015prototyping}. The latter, in
addition, would increase service utilization time in HPC systems, a critical
measure of efficiency. Considering the existing code base in scientific
computing, software infrastructures that allow a clean transition from legacy
systems to new Exascale platforms while preserving flexible prototyping
towards the future are central~\cite{hwu2015transitioning}.

This group attempts to tackle the latter challenge through the development of
novel software prototyping infrastructure with an emphasis at a higher degree
of abstraction. The group deems the ability to rapidly construct software artifacts that can be
trusted in terms of the methods from their design up, easily discarded when
wrong and extended when right at low human and computational cost to be one
aspect of scientific software sustainability. In essence,
a motto for scientific software prototyping is \textit{fail hard, fail fast}.

\subsubsection{Participants}

\begin{itemize}
  \item Santiago N\'u\~nez-Corrales <nunezco2@illinois.edu>
  \item Chris Sweet <chris.sweet@nd.edu>
  \item Steven Brandt <sbrandt@cct.lsu.edu>
  \item Dominic Orchard <d.a.orchard@kent.ac.uk>
\end{itemize}

\subsubsection{Working group objective}

The group's objective is development of a prototype tool that allows domain scientists to generate
simulation prototypes through a simple, yet expressive declarative
strongly-typed programming language close to equational expressions. The result
of that specification is both an executable artifact as well as a specification
for scientific programmers to later flesh out completely and adapt to particular
infrastructures. This tool will start in the domain of computational physics,
and (probably) extend to other domains, and the code will be open source.

The underlying hypothesis of this project is that a strongly typed, declarative
problem-specification language with adequate constructs, to be implemented
through stencils and algorithmic skeletons tied to numerical methods, captures a
 subset of useful and interesting problems in physics and other domains.

\subsubsection{Gap or challenge}

The gap addressed by the project is described by
following four statements:

\begin{enumerate}
  \item Domain specific scientists need to strike a balance between being
  productive in their science areas and having some coding skills.
  \item Being a research software engineer is a career in its own right which
  has limited depth in the science basis, especially while serving large
  communities.
  \item Collaboration in science is made (mostly informally) through natural
  language and mathematics, but both present lots of opportunity for potential
  ambiguity towards software development.
  \item An unambiguous (formal) language that sets a neutral ground is missing,
  capable of lifting many of the complexities of constructing useful (and
  correct) software artifacts that may later evolve into more elaborate codes.
\end{enumerate}

The latter enumeration of elements is sustained in the fact that scientific
software packages are digital representations of partial knowledge architectures
behind research processes.

\subsubsection{Relevant people and resources}

The project is mostly self-contained and only requires development time from
participants, as well as a Git repository (active). As for skills required, the
following distribution between project participants was identified:

\begin{description}
  \item[Steven] parsing expression grammars, prototype stencil
  \item[Santiago] functional programming, scientific programming
  \item[Dominic] semantics, types, compilers, languages
  \item[Chris] numerical methods
\end{description}

In terms of practice and experience, a sample scientific community is required
in order to work around the capabilities of the tool. In addition, use cases
are required (e.g., micromagnetics~\cite{fischbacher2007systematic}).

\subsubsection{Plans}

The group's general plan is to develop a functional (meta-)prototype before WSSSPE5, and
if useful, then develop a larger software infrastructure tied to
particular cyberinfrastructure resources.

\subsubsection{SMART steps}

\begin{enumerate}
  \item Define or choose the description language depending on the potential user
  base.
  \item Develop the language interpreter/compiler for the equational description
  language.
  \item Define a set of fundamental software patterns to translate to algorithmic
  skeletons.
  \item Implement a translation mechanism from the language to the patterns.
  \item Test the system with three known solvable problems and evaluate with a
  set of functional and non-functional metrics:
  \begin{enumerate}
    \item Wave equation
    \item Newtonian evolution
    \item Boussinesq equations
  \end{enumerate}
\end{enumerate}

\subsubsection{More information \& joining instructions}

S. Brandt has shared the following background links:

\begin{itemize}
  \item Piraha parser code: \url{https://github.com/stevenrbrandt/piraha-peg/blob/master/piraha.jar}
  \item A stripped down peg file (grammar) which describes a sequence of equations:
\url{https://www.cct.lsu.edu/~sbrandt/eqns.peg}
  \item An example equation file: \url{https://www.cct.lsu.edu/~sbrandt/wave.eq}
\end{itemize}

To join this project, please visit the GitHub repository at
\url{https://github.com/nunezco2/S2PI} or join the WSSSPE Slack channel
\texttt{\#wg-sci-soft-proto}.
