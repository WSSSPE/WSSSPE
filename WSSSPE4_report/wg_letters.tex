%%%%%%%%%%%%%%%%%%%%%%%%%%%%%%%%%%%%%%%%%%%%%%%%%%%%%%%%%%%%
\subsection{Letters of evaluation for computational scientists}
\label{sec:letters}
%%%%%%%%%%%%%%%%%%%%%%%%%%%%%%%%%%%%%%%%%%%%%%%%%%%%%%%%%%%%

%% See
%% https://docs.google.com/document/d/1prkkOddRNqIJ4n9qT0XeQIf5TJHWnESdoLFy6rUGL28/edit#heading=h.z17pqk0fi6d
%% for the document we drew up at the workshop

Letters of reference used for hiring, tenure, and promotion purposes
are typically first read by departmental committees that have
expertise in the core areas represented by the department. On the
other hand, computational scientists and in particular researchers
working on scientific software -- more or less by definition --
are typically interdisciplinary, and the achievements that are
assessed in such letters are consequently often meaningless or at
least hard to evaluate for disciplinary committees.

Both committees and letter writers need to be aware of this. Providing
best practice examples, and training both writers and recipients of
letters may be necessary to level the playing field for computational
scientists.

\subsubsection{Participants}

\begin{itemize}
\item Alice Allen <aallen@ascl.net>
\item Gabrielle Allen <gdallen@illinois.edu>
\item Wolfgang Bangerth <bangerth@colostate.edu>
\item Bruce Childers <childers@cs.pitt.edu>
\item Lorraine Hwang <ljhwang@ucdavis.edu>
\item Daniel S. Katz <d.katz@ieee.org>
\item Frank L{\"o}ffler <knarf@cct.lsu.edu>
\item Kyle Niemeyer <Kyle.Niemeyer@oregonstate.edu>
\item Janos Sallai <janos.sallai@vanderbilt.edu>
\end{itemize}

\subsubsection{Working group objective}

Scientists working on scientific software are often located in
disciplinary departments, depending on whether their software
originates from the mathematical, physical, chemical, or other
disciplines. As a consequence, they are frequently outside the core
areas of their science, and their contributions are typically to both
the research activity their software enables, as well as on
algorithm and implementation aspects. This presents issues when
letters of evaluation for hiring, tenure, and promotion do not specifically
cover how this is relevant to the discipline.%
\footnote{The extensive use of such letters, and the problems that are
  associated with it, may be an issue specific to the United States.}

This group believes that the authors of scientific software provide important
services to departments that are no less than strictly disciplinary
research. Consequently, leveling the playing field with more
disciplinary candidates for hires, tenure, and promotion requires that
letter writers be aware of how their letters will be read by
committees. It also requires that committees be aware that such
letters often look different and may provide a different perspective
of how a candidate's achievements should be assessed. For example, in
mathematics a typical candidate would be evaluated on the difficulty
and depth of the statements she may have proven in their papers. On
the other hand, an author of mathematical software would likely be
evaluated based on the impact of her software, or the number of citations of
the publications that describe it. She may also be evaluated by how
widely the software is used \textit{outside} mathematics, a criterion
that is rarely used for more disciplinary mathematicians.

\subsubsection{Gap or challenge}

Addressing this problem likely requires that letter writers pay
particular attention to who exactly the audience of such letters is,
and tailor the message by specifically highlighting how the work of a
candidate benefits the department and discipline that is considering a
candidate. On the other hand, committee members also need to pay
attention to the fact that there are areas that are important to the mission of
their department and professional community in which different
standards for evaluation hold.


\subsubsection{Relevant people and resources}

Affecting letter writers and readers essentially requires raising
awareness of the issue. Various organizations have attempted to do so
through letters to the community, adjusting definitions of what counts
in science, etc. Specifically, the group is aware of the following
resources:

\begin{itemize}
\item The Computing Research Association has produced a ``Best
  practices memo'' on the issue \cite{PSU99}. Among other points,
  the memo also clarified that conference proceedings are
  \textit{the} venue in Computer Science, and this has been used to
  convince deans and provosts of the value of these proceedings
  publications. The article may serve as a template for computational
  science.
\item The National Science Foundation has produced a number of ``Dear
  Colleague'' letter that consider the issue of software and, taken
  together, define a ``product'' for the purposes of
  2-page biosketches in a way that does not include only publications, but also
  software.%
  \footnote{The most pertinent such letter is NSF
      14-059, ``Dear Colleague Letter - Supporting Scientific
      Discovery through Norms and Practices for Software and Data
      Citation and Attribution''~\cite{nsf-dcl-citation}.}
  This definition is ultimately codified in the National Science
  Foundation's ``Proposal and Award Policies and Procedures Guide''
  (often abbreviated as GPG) which in 2013 renamed the
  ``Publications'' section of 2-page biosketches to ``Products'' (see
  Chapter II.C.2.f(i)(c)) and now explicitly allows software to be
  referenced in this section.
\item There is also a 1995 report by the National Academies about
  Computer Science that addresses some of these issues, specifically
  on how writing software should be considered in comparison to more
  theoretical research \cite{NRC-careers-94}.
\end{itemize}



\subsubsection{Plans}

There are essentially two important strategies that build on each
other: (i) raising awareness of the problem beyond just those who are
affected by it (namely, computational scientists working on scientific
software), and (ii) providing letter writers, letter readers, and
evaluating committees with guidance on what criteria are relevant in
assessing scientific software authors.

Concrete guidance---for example in the form of ``Dear Colleague''
letters as mentioned in the previous section---is most valuable if it
comes from respected bodies such as professional societies or
established and respected organizations. Getting these to act
will only be possible if the problem is widely acknowledged, and so
the first of the goals above should be the current focus.

The group will attempt to address it by organizing sessions at conferences
that address the career problem, as well as writing editorials that
can be published in the magazines of professional societies. These
editorials ought to outline best practices for letter writers that
specifically (i) make clear the contribution of a candidate to their
interdisciplinary area, (ii) the relevance to their home discipline, and
(iii) why writing software is good for the discipline itself.


\subsubsection{SMART steps}

The group has identified a few next steps, even if they may not satisfy the
exact criteria of the SMART system. Specifically:
\begin{itemize}
  \item Find and review the documents listed above, send out summary
    (Wolfgang Bangerth)
  \item Gather feedback from the other members of the group
  \item Write editorial for SIAM (Wolfgang)
  \item Write two emails to each other group members listed above about
    also writing editorials in their communities.
\end{itemize}
In the longer term, the group hopes to gather others in the home disciplines of its members to
let professional organizations weigh in.


\subsubsection{More information \& joining instructions}

The section on resources above lists a number of letters and reports
that are worth reading. Ultimately, building a community large enough
to affect change is important; contact Wolfgang Bangerth at
<bangerth@colostate.edu> if you are interested in helping.
