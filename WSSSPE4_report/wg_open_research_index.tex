%%%%%%%%%%%%%%%%%%%%%%%%%%%%%%%%%%%%%%%%%%%%%%%%%%%%%%%%%%%%
\subsection{Open research index}
\label{sec:open-research-index}
%%%%%%%%%%%%%%%%%%%%%%%%%%%%%%%%%%%%%%%%%%%%%%%%%%%%%%%%%%%%

%\note{Dan to write this}

The aim of this group is to investigate the building of an index of research products in an open sustainable manner.  Our goal is not to eliminate commercial products, but to build on what is there and provide data and services that are missing.

The Open Research Index should take in all research products (papers, software, datasets, workflows, etc.) from their publishers and recorders (journals, societies, domain repositories, government [open access] repositories, preprint servers, general repositories [e.g., figshare, zenodo]) and other services (CrossRef, ORCID).
Each product should list authors and citations and allow people to search the resulting network.
Users should also be able to interact with their own record and edit it, like Google Scholar allows.

\subsubsection{Participants}

\begin{itemize}
\item Gabrielle Allen <gdallen@illinois.edu>
\item Bruce Childers <childers@pitt.edu>
\item Robert Haines <robert.haines@manchester.ac.uk>
\item Caroline Jay <caroline.jay@manchester.ac.uk>
\item Daniel S. Katz <d.katz@ieee.org>
\item Robert McDonald <rhmcdona@indiana.edu>
\item Daniel Mosse <mosse@cs.pitt.edu>
\item Kyle Niemeyer <Kyle.Niemeyer@oregonstate.edu>
\end{itemize}

\subsubsection{Working group objective}

The working group's plans are relatively simple to express, though quite complex to undertake:

\begin{enumerate}
\item Determine a plan to build an open research index that allows various stakeholders to satisfy their needs
\item Then determine if the plan is feasible
\item If so, then obtain resources
\item If successful, then build the index
\end{enumerate}

\subsubsection{Gap or challenge}

Google and others provide some services now.
But these services (and the underlying data) could be removed at any time.
And the community cannot build new services.

\subsubsection{Relevant people and resources}

People/organizations who might be willing to contribute (or whose expertise is needed), and how they will contribute:

\begin{itemize}
\item ORCID, CrossRef, DataCite, ImpactStory/Depsy, \url{altmetrics.org}, Plum Analytics, GitHub, Open AIRE, CHORUS, FORCE11, COS, SHARE, CASRAI, Portico, \url{softwareheritage.org}, DBLP, eSTEP (CWO)
\item We could work with some initial publishers who don't have competing services (domain societies), PubMed
\item We need to discuss this idea with potential funders, and ask what data/services would they like to have?  And are they willing to invest in this?
\end{itemize}

The external resources that are needed are money and time; it is unclear how they can be brought in.

\subsubsection{Plans}

At the time of the meeting, and today as well, it is unclear who has the time and energy to pursue this idea.

If we identify a leader (PI), a plan could be:
\begin{enumerate}
\item Obtain funding for a set of initial discussions (or find a group who feel that this is important enough that they will take the time to do it anyhow)
\item Talk with potential partners (publishers, orgs, funders) (could be combined with meeting below)
\item Talk with the Google Scholar developer for better understanding of what they did and are planning in the future
\item Obtain funding for a meeting (perhaps from NSF, perhaps a Dagstuhl meeting)
\item Hold a community meeting to define a plan (which should be large enough to include representative of all stakeholder groups, and small enough that we still have a good open discussion)
\item Obtain funding to implement the plan, including contributions and commitments from the stakeholders
\end{enumerate}

Alternatively or additionally, we could build a mailing list for us and discuss further, depending on how receptive others are to this idea.  At WSSSPE4, there seemed to be enough interest to do this, so we set up a Slack channel within the WSSSPE team, called \texttt{\#wg-open-research-idx}. (See \S\ref{sec:slack} for instructions on how to join the Slack WSSSPE team.)

We also discussed some possible more detailed plans:

\begin{enumerate}
\item Look at curriculum lattes (Brazil) and/or Researchfish (UK).  They have some problems that people do not like, we could check what features they has and what the issues are.

\item Look at what CrossRef data is

\item What research products should we track?

\item What model should be used?
\begin{itemize}
    \item Crawling and obtaining data
    \item Consortium model
    \item Need analysis and decision
\end{itemize}

\item Need agreements with publishers, repositories, services
\begin{itemize}
    \item Some of these will have costs, though some may be free
    \item Need to determine costs
    \item Perhaps publishers would pay us to be listed eventually, as we gain power
    \item To some extent, we would be competing with some of the publishers? products
\end{itemize}

\item If crawling model:
\begin{itemize}
    \item How to crawl all of these products?
    \item What data (metadata) should be the output of the crawls? What happens when the products being crawled and/or their metadata change?
    \item How to store all of this data (metadata)?
\end{itemize}

\item What services to provide on this data (metadata)?

\item Who will create these services?

\item What services will the publishers allow to be provided?

\item Need to develop a plan that shows incremental progress and successes

\item Consider applying to Google project call to get Google interest and support?
\end{enumerate}


\subsubsection{SMART steps}

The working group discussed a number of SMART-like steps, though most of them did not fulfill all the SMART criteria, such as identifying who would do them and when:

\begin{enumerate}
\item Develop a community mailing list/forum (Kyle, during the workshop).  This was completed; see \S\ref{sec:wg-open-research-index-list}.

\item Find a PI (we suggest Neil Chue Hong)

\item Apply for initial funding. For the US, we discussed the Arnold Foundation Open Science award as a planning activity (who, due Dec 15), though without a PI, this seems unlikely to be done.
For the UK, we didn't have a suggestion.

\item Determine project name, initial vision, initial mission, logo (who, when)

\item Identify initial use cases (who, when)

\item Identify potential stakeholders and partners (who, when)

\item Discuss this idea with potential partners (who, when).
Then get their feedback \& buy-in.
While doing so, seek suggestions for the right project manager to coordinate this

\item Build up an advisory board (who, when)

\item Build up a leadership committee and determine a leader (who, when)

\item Hire a project manager (who, when)

\item Iterate use case with stakeholders (who, when),
e.g., funders, Researchfish, researchers ?

\item Iterate vision \& mission with leadership committee and advisory board (who, when)

\item Develop initial development plan (who, when)

\end{enumerate}


\subsubsection{More information \& joining instructions}\label{sec:wg-open-research-index-list}

If you are interested in more information, please join the WSSSPE team's Slack channel for this working group: \texttt{\#wg-open-research-idx}. (See \S\ref{sec:slack} for instructions on how to join the Slack WSSSPE team.)
